%!TEX TS-program = xelatex
%!TEX encoding = UTF-8 Unicode
%
% This template has been downloaded from:
% https://github.com/posquit0/Awesome-CV
%
% Author:
% Claud D. Park <posquit0.bj@gmail.com>
% http://www.posquit0.com
%
% Template license:
% CC BY-SA 4.0 (https://creativecommons.org/licenses/by-sa/4.0/)
%

% A4 paper size by default, use 'letterpaper' for US letter
\documentclass[12pt, a4paper]{awesome-cv}

% Configure page margins with geometry
\geometry{left=1.4cm, top=2cm, right=1.4cm, bottom=1.8cm, footskip=.5cm}

% Specify the location of the included fonts
\fontdir[fonts/]

% Color for highlights
% Awesome Colors: awesome-emerald, awesome-skyblue, awesome-red, awesome-pink, awesome-orange
%                 awesome-nephritis, awesome-concrete, awesome-darknight
\colorlet{awesome}{awesome-red}
% Uncomment if you would like to specify your own color
% \definecolor{awesome}{HTML}{CA63A8}

% Colors for text
% Uncomment if you would like to specify your own color
% \definecolor{darktext}{HTML}{414141}
% \definecolor{text}{HTML}{333333}
% \definecolor{graytext}{HTML}{5D5D5D}
% \definecolor{lighttext}{HTML}{999999}

% Set false if you don't want to highlight section with awesome color
\setbool{acvSectionColorHighlight}{true}

% If you would like to change the social information separator from a pipe (|) to something else
\renewcommand{\acvHeaderSocialSep}{\quad\textbar\quad}

\newcommand{\coloredhref}[3][blue]{\href{#2}{\color{purple}{#3}}}%


%-------------------------------------------------------------------------------
%	PERSONAL INFORMATION
%	Comment any of the lines below if they are not required
%-------------------------------------------------------------------------------
% Available options: circle|rectangle,edge/noedge,left/right
\photo{./pictures/SoumyaPass.jpg}
\name{Soumya}{Prasad}
\position{Kvalitetssäkring{\enskip\cdotp\enskip}CI pålitlighet}

\mobile{(+46) 73-340 12 29}
\email{soumya.saligram@gmail.com}
\github{chisma}
\linkedin{soumyaprasad}
% \gitlab{gitlab-id}
% \stackoverflow{SO-id}{SO-name}
% \reddit{reddit-id}
% \extrainfo{extra informations}

%-------------------------------------------------------------------------------
\begin{document}

% Print the header with above personal informations
% Give optional argument to change alignment(C: center, L: left, R: right)
\makecvheader

% Print the footer with 3 arguments(<left>, <center>, <right>)
% Leave any of these blank if they are not needed
\makecvfooter
  {\today}
  {Soumya Prasad~~~·~~~Resume}
  {\thepage}

\cvsection{Mål}
\begin{cvparagraph}

    Vil arbeta som test automatiserare för moderna web applikationer eller mobil appar samt med "Dev-ops" att säkerställa
    värje leverans som goras är helt regressionstestad. Strävar efter att utveckla kunskaper inom Containerization samt ser fram
    emot att börja bygga kompetens inom Google Cloud Platform och samhörande verktyger att skapa ett pålitlig Test-Ops miljö.
    Tycker om utmaningen att snabbt sätta mig in i ett intressanta problem.
\end{cvparagraph}

\cvsection{Profil}
\begin{cvparagraph}

    Erfaren och driven testare med helhetssyn. 10 års erfarenhet av test och kvalitetssäkring av
    clearing system för börser, offentliga appar för taxiversamhet, IoT sensornäterk och industriell automation produkter.
    Är analytisk, ivrig studerande, ansvarstagande och utåtriktad.
    Arbetar som Kvalitetssäkringsingenjör på IT avdelning av en berömd taxibolag sedan Maj 2019.
\end{cvparagraph}

\cvsection{Arbetslivserfarenhet}
  \begin{cventries}
    \cventry
      {Kvalitetssäkringsingenjör}
      {Cabonline Technologies AB}
      {Maj 2019 - pågående}
      {Stockholm}
      {
        \justify
        Cabonline Gruppen står bakom några av Nordens starkaste varumärken i taxiverksamheten(Tex: TaxiKurir, TopCab, Norges Taxi),
        Deras appar samlar 5700 bilar från flera av Nordens storsta taxibolag vilket minimerar väntetid oavsett var och när man vill åka,
        genom endast 3 enkla steger. Det betyder att appar anpassar sig värje "sprint", beroende på kundfeedback, betygsättning av föraren
        och aktuella marknads möjligheter. Det innebär i sin tur, flera fall av regression, på löpande sätt. Automatiserade end-to-end tester måste
        köras att godkänna leveransen.
        \hfill \break
        \hfill \break
        \begin{cvitems}
          \item Skriver end-to-end tester till alla webappar- interna och externa
          \item Skapar Docker lösningar till test miljöer samt integrering med CI pipeline
          \item Daglig övervakning av CI jobb för mislyckanden, flakyness
          \item Felsökning, debugging, optimering av Docker körning
          \item Uppdaterar, spårar backlog på JIRA
          \item Cypress, ReactJs, Docker, Jenkins, e2e testning, automatiserad visualla testning
        \end{cvitems}
      }

    \cventry
      {Test Utvecklare}
      {Cinnober Financial Technology AB/NASDAQ}
      {Sep 2017 - Maj 2019}
      {Stockholm}
      {
        \justify
        Cinnober utvecklade affärskritiska systemlösningar för börshandel, riskhantering och andra
        finansiella tjänster. Målgruppen för bolagets kärnverksamhet består främst av internationella börser,
        clearinghus, banker och mäklarhus. I "Johannesburg Stock Exchange" temet
        byggde vi ett nytt reltids clearinghus system- RTC, för Afrikas största börsen. Cinnober blev köpt av NASDAQ i 2019.
        \hfill \break
        \hfill \break
        \begin{cvitems}
          \item Ansvarig för automatiserade funktionella tester till REST APIet, inklusiva flera komplexa testfall inom riskhantering
          \item Samarbetade med systems utvecklare, för buggfixes och nya features, ibland med parallela leveranser
          \item Hade testerna integrerade i CI till flera workflows, så fort som möjlighet
          \item Övervakning, optimering av bygger
          \item Agerade som scrummästare för test-sprints
          \item Hanterade kod-coverage på CI
          \item JavaSE8, Eclipse, JUnit4, SmartGit, Python3, Confluence, JIRA, SonarCube
        \end{cvitems}
      }

    \cventry
      {Praktikant}
      {Yanzi Networks AB}
      {May 2017 - Jul 2017}
      {Stockholm}
      {
        \begin{cvitems}
          \item Testning av en websocker baserad API med Mocha/Chai biblioteket
          \item API Dokumentation med LaTeX, på Unix miljö
          \item JavaScript(ES6)
          \item Mocha/Chai/BDD
          \item REST API/websockets
          \end{cvitems}
      }


    \cventry
      {Associerad Ledare}
      {Schneider Electric}
      {Apr 2012- Maj 2013}
      {Bengaluru, Indien}
      {
        \begin{cvitems}
          \item Som en ämnesexpert, hållit på med att utveckla test-automatisering lösningen med TestComplete
          \item Arbetat med flera scrum grupper under acceptanstest och viktiga leveranser
          \end{cvitems}
      }

    \cventry
      {Senior Ingenjör}
      {Honeywell India Technology Labs}
      {Dec 2005 - Mar 2012}
      {Bengaluru, Indien}
      {
        \begin{cvitems}
        \item Definierade test planen och test specifikationer för integration och prestandatestning, manuell system validering
        \item Root cause analysis och löpande förbättringar(Kaizen)
        \end{cvitems}
      }

    \cventry
      {Projekt Ingenjör}
      {GE Intelligent Systems(Emerson India)}
      {Nov 2003 - Nov 2005}
      {Bengaluru, Indien}
      {
        \begin{cvitems}
        \item Läste designen samt tekniska specifikationer mot tagit av kunden och tabullera mjukvaru kräver samt BOM
        \item Utvecklade PLC logiken genom PLC ladder-språk och VB Script
        \item Deltog i kundinspektioner, projektrecensioner och mjukvaru demoer
        \end{cvitems}
      }
  \end{cventries}

%\newpage

\cvsection{Kurser}
  \begin{cventries}
    \cventry
    {Student}
      {Stanford University}
      {StanfordOnline: SOE.YDB-SQL0001}
      {Maj 2020 - ongoing}
      {
        \begin{cvitems}
        \item {Studerar i en certiferingskurs om Databaser}
        \end{cvitems}
      }

    \cventry
      {Student}
      {University of Pennsylvania}
      {coursera.org}
      {Feb 2020 - Apr 2020}
      {
        \begin{cvitems}
        \item {Klarade en certifieringskurs om Computational Thinking for Problem Solving}
        \end{cvitems}
      }

    \cventry
      {Student}
      {Lexicon IT Konsult}
      {Feb 2017 - Jun 2017}
      {Stockholm}
      {
        \begin{cvitems}
          \item JavaSE8 \& JavaEE fundamentals
          \item Spring/Hibernate fundamentals
          \item Maven fundamentals
          \item JUnit
          \item SQL fundamentals
          \item HTML5/CSS/JavaScript
          \item Eclipse IDE on Linux/Windows
           \end{cvitems}
      }
    \end{cventries}

    \cvsection{Utbildning}
    \begin{cventries}
    \cventry
        {Civilingenjörexamen}
        {Dayananda Sagar Institute of Engineering}
        {Bengaluru, Indien}
        {1999 - 2003}
        {
            \begin{cvitems}
            \item {Instrumenteringsteknik och blev godkänt med utmärkta betyg}
            \end{cvitems}
        }
  \end{cventries}

  \cvsection{Pris och ära}
  \begin{cvhonors}

    %---------------------------------------------------------
      \cvhonor
      {Honeywell}
      {Bravo-Gold:Process Excellence}
      {Bengaluru}
      {2009}

      \cvhonor
      {Honeywell}
      {ACS Special Award}
      {Bengaluru}
      {2007}

      \cvhonor
      {Honeywell}
      {Team excellence award}
      {Bengaluru}
      {2006}
  \end{cvhonors}
\end{document}

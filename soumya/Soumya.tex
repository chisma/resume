%!TEX TS-program = xelatex
%!TEX encoding = UTF-8 Unicode
%
% This template has been downloaded from:
% https://github.com/posquit0/Awesome-CV
%
% Author:
% Claud D. Park <posquit0.bj@gmail.com>
% http://www.posquit0.com
%
% Template license:
% CC BY-SA 4.0 (https://creativecommons.org/licenses/by-sa/4.0/)
%

% A4 paper size by default, use 'letterpaper' for US letter
\documentclass[11pt, a4paper]{awesome-cv}

% Configure page margins with geometry
\geometry{left=1.4cm, top=2cm, right=1.4cm, bottom=1.8cm, footskip=.5cm}

% Specify the location of the included fonts
\fontdir[fonts/]

% Color for highlights
% Awesome Colors: awesome-emerald, awesome-skyblue, awesome-red, awesome-pink, awesome-orange
%                 awesome-nephritis, awesome-concrete, awesome-darknight
\colorlet{awesome}{awesome-red}
% Uncomment if you would like to specify your own color
% \definecolor{awesome}{HTML}{CA63A8}

% Colors for text
% Uncomment if you would like to specify your own color
% \definecolor{darktext}{HTML}{414141}
% \definecolor{text}{HTML}{333333}
% \definecolor{graytext}{HTML}{5D5D5D}
% \definecolor{lighttext}{HTML}{999999}

% Set false if you don't want to highlight section with awesome color
\setbool{acvSectionColorHighlight}{true}

% If you would like to change the social information separator from a pipe (|) to something else
\renewcommand{\acvHeaderSocialSep}{\quad\textbar\quad}

\newcommand{\coloredhref}[3][blue]{\href{#2}{\color{purple}{#3}}}%


%-------------------------------------------------------------------------------
%	PERSONAL INFORMATION
%	Comment any of the lines below if they are not required
%-------------------------------------------------------------------------------
% Available options: circle|rectangle,edge/noedge,left/right
% \photo{./examples/profile.png}
\name{Soumya}{Prasad}
\position{QA Engineer}
% \address{Persikogatan 62, Hässelby, Stockholm, Sweden}

\mobile{(+46) 73-340 12 29}
\email{soumya.saligram@gmail.com}
\github{chisma}
\linkedin{soumyaprasad}
% \gitlab{gitlab-id}
% \stackoverflow{SO-id}{SO-name}
% \reddit{reddit-id}
% \extrainfo{extra informations}

%-------------------------------------------------------------------------------
\begin{document}

% Print the header with above personal informations
% Give optional argument to change alignment(C: center, L: left, R: right)
\makecvheader

% Print the footer with 3 arguments(<left>, <center>, <right>)
% Leave any of these blank if they are not needed
\makecvfooter
  {\today}
  {Soumya Prasad~~~·~~~Resume}
  {\thepage}

\cvsection{Summary}
\begin{cvparagraph}

  I am an engineer, with several years of experience within various areas of product testing. I enjoy automating tests, and in the process,
refine the product and its value, always, with the end user in focus. Continuously collaborating with developers, product owners,
account managers and testers - to best align with the use-case, while at the same time maintaining a high code coverage , are my strengths.

\hfill \break
I have been working as a QA Engineer at {\href{https://www.cabonline.com/}{\color{cyan}Cabonline Technologies AB}} since May 2019. I work
in the web team, hopping between various ReactJs applications. In my day job, I write end to end tests with {\href{https://www.cypress.io/}{\color{cyan}Cypress}},
dockerize the test environment and integrate into existing CI Pipelines. I also work with {\href{https://storybook.js.org/}{\color{cyan}Storybook}} \& visual component 
tests using {\href{https://loki.js.org/}{\color{cyan}Loki}}

\hfill \break
I am familiar with post trade solutions and
trading platforms in general(With hands-on knowledge of clearing technology for stock exchanges). I have also briefly worked
with IoT sensor networks and for several years in industrial automation.

\hfill \break
I am looking for testing roles involving UI testing of web/mobile services/products. I am easy to work with,
a keen learner, very communicative and thrive working in a team. Recently, I completed a course in Computational Thinking from University Of Pennsylvania.
I am very keen to start on projects where I could build my "TestOps" skills by deploying code-less UI tests over Google Kubernates Engine. 
I can communicate in English and Swedish.

\end{cvparagraph}

\cvsection{Experience}
  \begin{cventries}

    \cventry
      {QA Engineer}
      {Cabonline Technologies AB}
      {May 2019 - Current}
      {Stockholm}
      {
        \begin{cvitems}
          \item Write end-to-end tests for various web products
          \item Write component level visual tests to enable maximum UI coverage
          \item Dockerize the test environment for integration into CI pipelines
          \item Daily monitoring of CI builds for breakages, flaky tests
          \item Troubleshooting, debugging, optimization of CI/Docker bottlenecks and failures from the build log files
          \item Update, track backlog on JIRA
          \item Cypress, ReactJs, Docker, Jenkins, Loki, e2e testing, visual component testing
        \end{cvitems}
      }

    \cventry
      {Test Developer}
      {Cinnober Financial Technology AB/NASDAQ}
      {Sep 2017 - May 2019}
      {Stockholm}
      {
        \begin{cvitems}
          \item Writing functional tests for REST APIs
          \item Collaborate with developers in the team, for bug fixes or tasks to be automated
          for the current release and/or patch delivery
          \item Have the tests integrated to CI pipeline/s for various work flows, as soon as possible
          \item Daily monitoring of CI builds for breakages, flaky tests.
          \item Troubleshooting, debugging of CI bottlenecks and failures from the build log files
          \item Update, track backlog on JIRA
          \item Act as scrum master for testing sprints
          \item Drive consistent good performance on code quality jobs on CI
          \item JavaSE8, Eclipse, JUnit4, SmartGit, Python3, Confluence, JIRA, SonarCube
        \end{cvitems}
      }

    \cventry
      {Intern}
      {Yanzi Networks AB}
      {May 2017 - Jul 2017}
      {Stockholm}
      {
        \begin{cvitems}
          \item Testing of a websocket based REST API with Mocha/Chai libraries
          \item Documenting the API using LaTeX, in a unix environment
          \item JavaScript(ES6)
          \item Mocha/Chai/BDD
          \item REST API/websockets
          \item LaTeX
          \end{cvitems}
      }


    \cventry
      {Associate Lead}
      {Schneider Electric}
      {Apr 2012- May 2013}
      {Bengaluru, India}
      {
        \begin{cvitems}
          \item Worked on improving test automation with TestComplete.
          \item Worked closely with various scrum teams during user acceptance and release
          \end{cvitems}
      }

    \cventry
      {Senior Engineer}
      {Honeywell India Technology Labs}
      {Dec 2005 - Mar 2012}
      {Bengaluru, India}
      {
        \begin{cvitems}
        \item Define test plans and test specifications for integration, functional and performance testing, execution of test cases
        \item Root cause analysis and continuous process improvement(Kaizen)
        \end{cvitems}
      }

    \cventry
      {Project Engineer}
      {GE Intelligent Systems(Emerson India)}
      {Nov 2003 - Nov 2005}
      {Bengaluru, India}
      {
        \begin{cvitems}
        \item Study the design and technical specifications received from customer and understand the process requirements and deliverables
        \item Develop the main PLC logic using the Ladder language and VB scripting
        \item Participate in customer interactions, project reviews and software inspection
        \end{cvitems}
      }
  \end{cventries}

%\newpage

\cvsection{Education}
  \begin{cventries}
    \cventry
    {Student}
      {Stanford University}
      {StanfordOnline: SOE.YDB-SQL0001}
      {May 2020 - }
      {
        \begin{cvitems}
        \item {Studying a certification course on Databases}
        \end{cvitems}
      }
      \cventry
      {Student}
      {University of Pennsylvania}
      {coursera.org}
      {Feb 2020 - Apr 2020}
      {
        \begin{cvitems}
        \item {Studied a certification course on Computational Thinking for Problem Solving}
        \end{cvitems}
      }

      \cventry
      {Student}
      {Lexicon IT Konsult}
      {Feb 2017 - Jun 2017}
      {Stockholm}
      {
        \begin{cvitems}
          \item JavaSE8
          \item JavaEE fundamentals
          \item Spring/Hibernate fundamentals
          \item Maven fundamentals
          \item JUnit
          \item SQL fundamentals
          \item HTML5/CSS/JavaScript
          \item Eclipse IDE on Linux/Windows
           \end{cvitems}
      }

    \cventry
      {Bachelor of Engineering}
      {Dayananda Sagar Institute of Engineering}
      {Bengaluru, India}
      {1999 - 2003}
      {
        \begin{cvitems}
        \item {Studied Instrumentation engineering and completed course with excellent grades.}
        \end{cvitems}
      }
  \end{cventries}

  \cvsection{Awards}
  \begin{cvhonors}

    %---------------------------------------------------------
      \cvhonor
      {Honeywell}
      {Bravo-Gold:Process Excellence}
      {Bengaluru}
      {2009}
      
      \cvhonor
      {Honeywell}
      {ACS Special Award}
      {Bengaluru}
      {2007}

      \cvhonor
      {Honeywell}
      {Team excellence award}
      {Bengaluru}
      {2006}
      %  \item {2007}{ACS Special Award, Honeywell}
      %  \item {2006}{Team excellence award, Honeywell}
  \end{cvhonors}

\end{document}
